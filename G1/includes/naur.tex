\section{Naurs synspunkter}
Her vil vi samlet redegøre for de mest relevante ideer og holdninger, Peter Naur
fremsætter i de udleverede artikler.

\subsection{Computere og Databehandlingsapparater}
Naur har altid taget stor afstand til ordene ``Computer Science'' og ``Computer
Scientist'', hvilket man kunne tænke sig skyldtes hans holdning om, at et
sådant kursus burde beskæftige sig mere med data, samt behandling, modellering,
repræsentationer og strukturer heraf.\\
`Datalogi' (eng. datalogy) er under disse forudsætninger, et langt mere sigende
og repræsentativt ordvalg, der understreger Naurs ønskede hovedfokus for en
sådan videnskab.
Han anerkender dog samtidig, at der (på hans tid) er/vil blive behov for folk
med en større praktisk viden og foreslår efter NASA konferencen en opdeling.\\
I Danmark er denne opdeling af disciplinerne blevet en realitet med indførslen
af blandt andet datamatiker- og software ingeniør-uddannelserne, som har sit
hovedfokus i de konkrete kodesprog, samt anvendelsen heraf.\\
Det står i kontrast til datalogien, hvor man forsøger at give de studerende et
bredt indblik i/overblik over de discipliner, der er involveret i produktionen
af et nyt system.


\subsection{Uddannelse}
Der er nu kommet et Datalogisk Institut ved Københavns Universitet (DIKU). Det
har imidlertid ikke afstedekommet den udvikling, Naur havde håbet og foreslået.
Der er stadig ikke en elementær datalogisk undervisning i folkeskolen, omend
langt størstedelen af danske skoleelever har adgang til en datamat - enten på
skolen eller hjemme. Samtidig bliver Folkeskolens Afgangsprøve (FSA) i 9. klasse
stadigt mere digitaliseret.

Den teknologiske udvikling har muliggjort dette i form af kraftigere
datamater, samt mere brugervenlige/-orienterede applikationer og GUI'er.\\
Det er dermed ikke længere en problemstilling i vores samfund, at børn og unge
ikke har den fornødne viden om datamaskiner, som Naur ønskede at introducere

Peter Naur udtrykker et ønske om, at den nye uddannelse - datalogi - har et
skarpt fokus på gruppearbejde og projektarbejde, idet sådanne arbejdsformer
underbygger forståelsen for, hvad det vil sige at skulle udvikle et
produktionssystem [Rosenkjærforedrag, 1967], omend i mindre skala.

Igennem Rosenkjærforedraget formidler Naur sin vision om en datalogi, der er
præget af åbenhed omkring og fokus på ideerne og teorien, fremfor værktøjerne.
Naur fremfører altså pro-kreativiteten og pro-`den selvstændige tanke' blandt
datalogerne og deres virke, som kunne være affødt af hans tid som astronom.
Som astronom har man jo netop forpligtigelsen til at tænke nyskabende og
anderledes, hvilket kunne underbygge ovenstående.
