\section{introduction}


\section{Redegørelse}
Den nuværende videnskabelige metode går ud på at opsætte en hypotese, udføre eksperimenter, observere hvad der sker og bagefter sammenholde resultaterne med dem fra hypotesen. Ud fra det kan man verificere eller falsificere sin hypotese og så det kan man enten opstille en generel teori over emnet eller man kan revidere sin teori, alt efter, i hvor stor grad, ens resultater passer til hypotesen. 

\section{End of Theory beskrivlse}
End of Theory er en artikel, der beskriver en ny tendens inden for videnskab. Den handler om at analysere store mængder data med matematik og fra de resultater man har fået så drager man sine konklusioner. Denne type "forskning" adskiller sig fra den normale praksis ved at den ikke bruger modeller og at man ikke behøver at opstille teorier. Man behøver ikke at vide hvorfor tingene sker, man ved bare at de sker. Denne praksis er muliggjort på grund af den store mængde data der bliver opsamlet af blandt andet Google og de store petabyte lagere. Data er ikke længere noget man visualisere, men er i stedet noget man ser igennem matematik hvormed korrelationen bliver tydelig.

