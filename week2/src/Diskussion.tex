\section{Diskussion}
Artiklen fokuserer meget på at vi, ved hjælp af data, ikke længere har brug for modeller. Dette er fordi vi kan analysere så meget data at vi ikke længere behøver at opstille teorier om hvorfor tingene sker. Det, han siger, er med andre ord at vi fra nu af skal analysere os frem til en korrelation i stedet for, at finde en kausalitet. 

Dette er i stærk kontrast til den filosofi der er nu, hvor man fokuserer meget på kausalitet. For at sætte hans synspunkt i perspektiv, så er realisme og anti-realisme et stort diskussionspunkt, fordi de sætter spørgsmålstegn ved hvordan man skal lave videnskab, på samme måde som den fremgangsmåde artiklen henviser til, men hvor realisme og anti-realisme aldrig har været enige om hvordan man skal forske, så er de enige om at der skal opstilles modeller og teorier. Det er modsat artiklens fremgangsmåde, der forkaster modellerne og gør teorierne meningsløse, da de ikke længere bliver brugt til noget, i kraft af at mængden af data er det afgørende. \\
Det er en interessant betragtning at en ny fremgangsmåde i nogle tilfælde kan fuldstændigt overflødiggøre en stor del af videnskabens historie. Alle diskussioner omkring kausalitet lyder meningsløse, hvis en fremgangsmåde hvor kausalitet ikke eksisterer bliver den mest anvendte. 

En sådanne situation, hvor videnskabens virkelighed bliver slået i stykker minder på mange måder om Kuhn's paradigmeskift, hvor man påpeger den gamle virkeligheds fejl og en ny virkelighed begynder at tage form indtil videnskabsmændene bliver overbeviste om, at de har modtaget nok beviser for den nye tilgang til videnskab, til at den kan blive den nye virkelighedsopfattelse. \\
Dog er der nogle huller i den nye fremgangsmåde. Den er god når vi allerede har en ide om hvad skal kigge efter, men hvis man bare har en masse data, så kan det være svært at vide hvilke sammenhænge der er relevante, hvis man overhovedet har et mål til at starte med. Derfor er dataanalysen bedst, når man bruger den i forlængelse af den nuværende fremgangsmåde, i stedet for at gøre som artiklen foreslår, hvor man erstatter den ene fremgangsmåde med den anden. 

Grunden til at den ikke helt kan erstatte den nuværende fremgangsmåde er, at den ikke prøver at forklare de ting som den nuværende opfattelse ikke kan forklare, hvilket går at den ikke kan forvise den anden form. \\
Det er også fordi korrelation ikke er nok for en videnskabsmand. Hvis vi stillede os tilfredse med korrelation, så ville det være en tilbagegang, der forhindrer nye store opdagelser. Hvis vi kigger historisk så er søgen efter kausalitet det, der har rykket verden mest. 

Et eksempel ville være tyngdekraften. Den blev opdagede af Newton, der ville undersøge hvorfor ting faldt til jorden. Hvis Newton havde store mængder data og skippede hvorfor-spørgsmålet, så ville vi ikke have den teoretiske viden om tyngdekraften, der har gjort det muligt at flyve, beregne kastebaner eller sende de satellitter op i rummet, som har været essentielle får overhovedet at kunne indsamle så meget data. 

Et andet eksempel på hvorfor det er dårligt kun at kigge på korrelation, kunne være fugletræk og kulde. Fugle trækker syd på når temperaturen falder. Denne korrelation fortæller intet om hvorfor fugle trækker syd på, men hvis man leder efter kausaliteten vil man finde svaret.