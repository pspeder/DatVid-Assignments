\section{Modeller}
Vi vil her kort beskrive forskellen imellem den videnskabelige verdens
forståelse af en model og Andersons udlægning af, og specielt hans sprogbrug
omkring, denne.

\subsection{Hvad er en model?}
Wikipediaartiklen om videnskabelige modeller siger om emnet, det følgende:
\begin{quote}
Scientific modelling is a scientific activity the aim of which is to make a
particular part or feature of the world easier to understand, define, quantify, visualize, or simulate.\footnote{Kilde: {\url{http://en.wikipedia.org/wiki/Scientific_modelling}} ll. 1-3}
\end{quote}

Under denne definition er Andersons sprogbrug omkring modeller forkert, idet
hans forslag til, hvordan videnskab skal behandles, netop stræber efter ikke at
søge en ``\ldots particular part or feature of the world\ldots.''\\
Hans filosofi går netop ud på, ikke at have en forudindtaget opfattelse omkring
verden og om det, man ønsker at undersøge (hvis man da ved dét).

\subsection{Hvorfor modelleres der i videnskaben?}
Der har tidligere været et behov for måder at repræsentere virkeligheden på.
Modeller giver videnskabsmanden muligheden for at afprøve sin hypotese, selv
under omstændigheder, hvor direkte måling og eksperimentering er umulig.\\
Sådanne områder inkluderer blandt andre astronomien og har tidligere inkluderet
mange flere.

Modeller var så at sige, den metode, der mest konsistent tilnærmelsesvis beskrev
verden.

Vi tolker Chris Anderson-artiklen, som et forsøg på at formidle 
Nu til dags, som altid, er modeller et rigtig godt værktøj i
filosofikerens/videnskabsmandens værktøjskasse, til at beskrive både deres
forsøg.

