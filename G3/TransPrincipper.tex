\begin{document}

\subsection{Transistor-Baserede Computere}
En transistor-baseret computer er en computer som har en mængde kerner i 
sig til at lave beregninger. Ved brug af transistorer som har to tilstande, 
(slukket eller tændt), kan man foretage komplicerede beregninger. Jo 
flere transistorer der ligger på en chip eller kerne, jo mere større 
beregninger kan foretages. Samtidigt er en anden vigtig aspekt af 
transistor-baserede computere at beregninger kan parelleliserers over 
flere kerner, hvilket effektiviserer mængden af beregninger per 
takt.
\end{document}
