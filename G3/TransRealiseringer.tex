\begin{document}
\subsection*{Transistor-Baserede Computer}
\subsubsection*{Mulige Teoretiske Performance}
I teorien, ifølge Moore's lov, ville en CPUs taktfrekvens 
fordoble hver 18. måned, da transistorerne ville blive halvt så store 
som før, hvilket ville lade en computer skifte tilstand på den halve 
tid. Dette ville lade en transistor-baseret computers performance 
blive uendelig stor da de hele tiden ville forbedre sig. Dog er dette, 
i praksis, umulig; da computerne skal trække mere strøm, ville noget 
af dette strøm blive omdannet til varme grundet resistens, med den 
betydning at computerne ville overophede betydeligt hurtigere.
\subsubsection*{Evne Som Universel Computer}
En transistor-baseret computers evne som en universel computer er 
meget høj - transistorer har en meget lav pris per transistor som bliver 
ved med at blive billigere, hvilket har betydet at ting som CPU og RAM 
er blevet billige at producere, og derved også billige at købe.
\subsubsection*{Softwaren}
I en transistor-baseret computer, skal softwaren kunne udnytte 
\begin{itemize}
\item En eller flere kerner,
\item Korrekt brug af RAM, og
\item Parallelisering
\end{itemize}
i en transistor-baseret computer. Ud over dette, da mange computere 
har en eller anden form for GPU, skal softwaren også kunne finde ud af 
at bruge GPUen korrekt, så softwaren ikke overbelaster CPUen.
\end{document}
