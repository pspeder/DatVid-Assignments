\section{Introduktion}
Vi vil i denne opgave beskrive tre forskellige fysiske realiseringer af en computer. Den transistor-baseret computer, den biologiske computer og kvantecomputeren. Vi vil i opgaven komme ind på deres teoretiske muligheder, skrive lidt om deres evne som universel computer og komme med bud på hvad softwaren i computeren kunne være. De tre computere er omtalt i artiklerne {\it Supercomputere - mange bække små} af Brian Vinter (2010), {\it Cramming more components onto smaller circuits} af Gordon Moore (1965), {\it Computing with DNA} af Leonard Adleman (1998) og {\it Quantum computing} af Amit Hagar (2011).
