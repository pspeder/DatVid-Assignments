
\section{DNA-Computere}

\subsection{Fysiske realiseringer}
Idag er forskning indenfor DNA-computere langsomt begyndt at bære frugt. Vi
er begyndt at kunne implementere logiske gates, og programmere disse.

Meget af den viden, vi har om DNA-computere, stammer fra Adlemans forskning i
feltet, og hans udgivelse i 1994, hvori han beskriver sin syv-punkters
Hamilton-problem-løser til fulde.

Det har vist sig at DNA-computere har et langt større anvendelsesgrundlag end
først antaget; dels grundet deres potentiale til at kunne benyttes i medicinske
henseender, og dels på grund af det potentiale, der ligger i parallelisering
på en sådan skala. 

\subsection{Relation til Turing-maskinen}
Hvis det viser sig nemmere at lave algoritmer til DNA-computeren, har den
potetialet til at kunne løse NP-hårde problemer ved at udnytte de mange tråde
en sådan maskine består af.

Det vil ikke blot betyde et kæmpe løft i hvordan vi ser på computere og
hvad de kan i dag, men kunne risikere at blive en kæmpe sikkerhedstrussel.\\
Sådanne computere ville kunne dekryptere alle enkrypteringer, vi i dag anser
som ubrydelige.

Det er dog tvivlsomt, at vi vil se en DNA-computer med så stor beregningskapacitet
i den nære fremtid, af den simple årsag, at de er dyre at producere, og der vil
være en stejl indlæringskurve, før vi fuldt forstår dem.

Af samme årsag (at disse computere konceptuelt er meget anderledes end hvad
vi, som mennesker, er vant til), vil disse computere ikke gøre sig godt som
``hverdagscomputere'' --- måske med undtagelse af hardcore-gamere, der ville
kunne have god udnyttelse af en DNA-GPU (hvis det viser sig muligt at
programmere).

Molekylærbiologier er nu begyndt at implementere logiske gates, sådan at
disse computere kan programmeres, (næsten) som vi gør det med transistor-baserede
computere. Der er dog stadig lang vej endnu til at finde optimale ``interne algoritmer''
i computerne.

Det betyder også, at vi har fået vores første kryds-og-tværs-løser på en
DNA-computer, omend denne kører med en ekstern computer, der omslutter
dens funktionalitet.

