
\section{DNA-Computere}

\subsection{Fysiske realiseringer}
Idag er forskning indenfor DNA-computere langsomt begyndt at bære frugt. Vi
er begyndt at kunne implementere logiske gates, og programmere disse.

Meget af den viden, vi har om DNA-computere, stammer fra Adlemans forskning i
feltet, og hans udgivelse i 1994, hvori han beskriver sin syv-punkters
Hamilton-problem-løser til fulde.

Det har vist sig at DNA-computere har et langt større anvendelsesgrundlag end
først antaget; dels grundet deres potentiale til at kunne benyttes i medicinske
henseender, og dels på grund af det potentiale, der ligger i parallelisering
på en sådan skala. 

\subsection{Relation til Turing-maskinen}
Hvis det viser sig nemmere at lave algoritmer til DNA-computeren, har den
potetialet til at kunne løse NP-hårde problemer ved at udnytte de mange tråde
en sådan maskine består af.

Det vil ikke blot betyde et kæmpe løft i hvordan vi ser på computere og
hvad de kan i dag, men kunne risikere at blive en kæmpe sikkerhedstrussel.\\
Sådanne computere ville kunne dekryptere alle enkrypteringer, vi i dag anser
som ubrydelige.

Det er dog tvivlsomt, at vi vil se en DNA-computer med så stor beregningskapacitet
i den nære fremtid, af den simple årsag, at de er dyre at producere.



Opgave B
Redegør for de tre fysiske realiseringers relation til Turing-maskinen, herunder drøft
1) maskinernes mulige teoretiske performance 2) deres evne som universel computer,
og 3) hvad der udgør softwaren i de tre fysiske realiseringer.

As of today, only partially functioning (or fully functioning - but with very
limited/varying perfomance) DNA systems have been implemented and realised.

We have yet to see the advances that we're seeing in the Quantum-computing
field, for example, where IBM says that Quantum computers for personal use might
only be 10 years away\footnote{http://www.extremetech.com/extreme/120229-ibm-shows-off-quantum-computing-breakthroughs-says-qubit-computers-are-close}.


