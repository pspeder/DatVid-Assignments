\section{Afd�kning af forskelle og ligheder}

Ligheder
\begin{itemize}
\item Faget har en bred r�kke af applikationer, der kan bruges med god fornuft i en andre discipliner.
\item 
\end{itemize}

Forskelle
\begin{itemize}
\item 
\end{itemize}

Tv�rfagligt felt. B�r v�re hj�lpefag til andre fag samt at have en specialisering.
Datalogen skal ikke fokusere p� selve programmet, men p� teorien bag. ACM siger at man skal v�re omstillingsparat.
Fokus p� processer og strukturer. Naurs hovedpunkter passer godt til det der tager lang tid i tier-core 1.
Begge projektorienteret.
Det skal kunne formidles.
Menneskers tilgang til programmet.
Skal hele tiden l�re mere.
Skal arbejde sammen med andre fagdiscipliner.
Datalogi kan ikke meget selv men er et v�rkt�j til at g�re meget.
Begge er tidssvarende.
Skal f�lge med udviklingen i samfundet.
CCS siger mindst et st�rre projekt.
Naur mener man skal opbygge sin egen teori omkring programmering.
B�r l�re principper frem for sprog. Forst�elsesfag.


Projektledelse/kommunikation i universitetet vs l�re om det i folkeskolen.
Mastery vs Data
Naur ligger mere fokus p� teori
CSS ligger v�gt p� nuv�rende uddannelse mens Naur kigger mod fremtiden.
Naurs tekst er idealistist mens CSS er pragmatisk
People/problems/tools vs 
CSS introducerer introduktionskurser til selv non-major. Naur mener ikke at programmering i folkeskolen er n�dvendigt, men vil have datalogiske kompetancer i folkeskolen.
Data er en repr�sentation af information. Datamaten er kun et v�rkt�j.
Udviklingen er g�et modsat hvad Naur �nskede. Mere computer, mindre computing
CSS fokuserer p� software, Naur p� data
Mindske gr�ften ved at uddanne, mens ?teori? kun gavner specialister

