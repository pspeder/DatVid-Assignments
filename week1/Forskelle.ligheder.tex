\section{Afdækning af forskelle og ligheder}

Ligheder
\begin{itemize}
\item Faget har en bred række af applikationer, der kan bruges med god fornuft i en andre discipliner.
\item 
\end{itemize}

Forskelle
\begin{itemize}
\item 
\end{itemize}

Tværfagligt felt. Bør være hjælpefag til andre fag samt at have en specialisering.
Datalogen skal ikke fokusere på selve programmet, men på teorien bag. ACM siger at man skal være omstillingsparat.
Fokus på processer og strukturer. Naurs hovedpunkter passer godt til det der tager lang tid i tier-core 1.
Begge projektorienteret.
Det skal kunne formidles.
Menneskers tilgang til programmet.
Skal hele tiden lære mere.
Skal arbejde sammen med andre fagdiscipliner.
Datalogi kan ikke meget selv men er et værktøj til at gøre meget.
Begge er tidssvarende.
Skal følge med udviklingen i samfundet.
CCS siger mindst et større projekt.
Naur mener man skal opbygge sin egen teori omkring programmering.
Bør lære principper frem for sprog. Forståelsesfag.


Projektledelse/kommunikation i universitetet vs lære om det i folkeskolen.
Mastery vs Data
Naur ligger mere fokus på teori
CSS ligger vægt på nuværende uddannelse mens Naur kigger mod fremtiden.
Naurs tekst er idealistist mens CSS er pragmatisk
People/problems/tools vs 
CSS introducerer introduktionskurser til selv non-major. Naur mener ikke at programmering i folkeskolen er nødvendigt, men vil have datalogiske kompetancer i folkeskolen.
Data er en repræsentation af information. Datamaten er kun et værktøj.
Udviklingen er gået modsat hvad Naur ønskede. Mere computer, mindre computing
CSS fokuserer på software, Naur på data
Mindske grøften ved at uddanne, mens ?teori? kun gavner specialister

