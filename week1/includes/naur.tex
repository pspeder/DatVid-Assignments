\section{Naurs synspunkter}
Her vil vi kort redegøre for de - i vores øjne - mest relevante/betydende
ideer og holdninger, der fremsættes i [1,2,3]

\subsection{}
Naur har altid taget stor afstand til titlen Computer Scientist og Computer Science generelt,
hvilket man kunne tænke sig skyldtes hans holdning om at et sådant kursus burde beskæftige sig mere
med data og behandling/modellering/repræsentation/strukturer heraf, samt de involverede processer.\\
Ordet datalogi (datalogy) er under
disse forudsætninger, et langt mere sigende og repræsentativt ordvalg, der understreger
Naurs ønskede hovedfokus for en sådan videnskab. Han anerkender dog samtidig, at der (på hans tid) er/vil blive behov
for folk med en større praktisk viden og foreslår efter NASA konferencen en opdeling.

I Danmark er denne opdeling af disciplinerne blevet en realitet med indførslen af datamatikeruddannelsen,
som har sit hovedfokus i de konkrete kodesprog, samt anvendelsen heraf. Det står i kontrast til datalogien,
hvor man forsøger at give de studerende et bredt indblik i/overblik over langt de fleste processer involveret i
"software engineering".

