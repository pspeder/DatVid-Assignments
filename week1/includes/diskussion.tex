\section{Diskussion}
Peter Naurs vision for datalogi er meget i overensstemmelse med det nuværende syn på datalogi. Naur ser tidligt at datalogi vil være meget anvendeligt i alle fag og kan blive et værktøj på linje med fx matematik. Vi kan nu se at det bliver brugt overalt, men det bliver ikke brugt på samme niveau som fx matematik. Det er blevet et specialiseringsfag på universitetsniveau og den almene befolkning kan ikke drage nytte af faget, men er i stedet afhængig af specialisterne. Naur foreslog at principperne bag datalogi blev lært på folkeskoleniveau, så alle kunne forstå teorien bag den computeren, men det er ikke sket endnu. ACM sætter fokus på universitetsuddannelsen, hvilket er okay, men de kunne brede sig mere og introducere datalogi til børn og unge. Vi lever i en verden med computere overalt, men kun de færreste af os ved hvordan de fungerer.

Både Naur og ACM er klar over vigtigheden i projektarbejde, der tager udgangspunkt i et erhverv. Dette er fordi programmeringserfaring er den bedste måde at få erfaring med datalogi. At programmere en problemstilling kræver en god teoretisk forståelse for problemet og hvilket gør faget velegnet til projektarbejde. Man skal dog havde et teoretisk grundlag, så man kan forstå den teori. Naur fandt tidligt ud af hvilke dele af datalogien, der var kernestof. Han mente at data, dataprocesser og datastrukturer var de vigtigste begreber, hvilket udviklingen har understøttet. ACM har lagt meget vægt på områder af datalogien hvor man lærer det teoretiske grundlag for disse tre begreber.

ACM ligger mere fokus på software end Naur gør. Naur fokuserer mere på forståelsen af metoderne bag softwaren, fordi software kan blive forældet, men problemstillingen vil være den samme og ved at fokusere på forståelsen, så er det nemmere at formidle den videre til andre. Det er et andet punkt hvor Naur ønsker mere fokus. Formidling mellem mennesker er ikke så højt prioriteret i ACM som hos Naur og interaktion mellem menneske og maskine er heller ikke højt prioriteret. Denne lave prioritering af kommunikation er yderst skadelig for forståelsen mellem datalog specialisterne og det almene folk. Mange af Naurs forslag gik ud på at mindske grøften mellem fagfolk og den almene bruger, men det er et punkt hvor ACM ikke bruger meget energi og der er ikke udsigt til de store ændringer på det felt.

Peter Naur har været en meget fremsynet mand, der hurtigt så mulighederne i datalogi. Mange af hans opfattelser er nu blevet en del af den grundlæggende arbejdsform i datalogi. Projekter og teoretisk forståelse er blevet standard og selvom pensummet ikke lægger op til erhvervserfaring, så er det en central del af en datalogs indlæringsproces og det pensum som ACM har lavet, giver giver rig mulighed for at koble erhvervserfaring på forløbet.