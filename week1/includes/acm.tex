\section{ACM}
ACMs id\'{e} af hvad en computer scientist er fokuserer på nogle træk som 
er dybt uenige med Naurs definition af en datalog. Men hvad både ACM 
og Naur var enige om var at en computer scientist/datalog, (hermed skrevet som datalog), skal være fleksibel. En datalog skulle kunne tackle en opgave 
ud fra den læren og erfarenhed datalogen har haft fra sin uddannelse og 
professionale liv. Derfra dog var de uenige.\\
Hvor Naurs fokus var på den teoretiske del, hvor han argumenterede for 
at man ikke behøvede en datamat til at lære og afprøve, mener 
ACM at en lige teoretisk og praktisk uddannelse er vigtig for at kunne 
være en successfuld computer scientist. De mente at ved at have en bred fane af læren, og villighed til at lære ny ting, ville man kunne være en dygtig 
computer scientist. Som resultat af dette, er deres fokus på at der skulle 
være fleksibilitet for en computer scientist på alle måder, både under 
oplæringen som også i det professionale liv.\\
ACMs tre hovedpunkter som de definerer som at være `three levels of 
mastery' er familiarity, usage og assessment. Deres grundlag for dette er at 
når man har en baggrund som består af både det grundlæggende og en 
bred vifte af forskellige, mere specialiserede faglig viden, ville man kunne 
nå disse tre hovedpunkter.
