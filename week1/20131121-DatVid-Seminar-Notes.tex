

\section{Generelt}
I de tidlige tekster forsøger man, at definere datalogien i højere grad end den forsøger at undervise i faget.

Programming as theory building involverer at man er nødt til at sætte sig ind i systemet og forstå problemet - ved at
kode.



\section{Projektledelse/Kommunikation}
Peter Naur udtrykker et ønske om en større udbredelse af datalogien, allerede
meget tidligt i uddannelsessystemet (folkeskolen). Dette står i skarp kontrast
til den udvikling datalogi som videnskab har gennemgået siden Naur fremførte sine
holdninger.\\
I ACM, såvel som i idags samfundet generelt, lægges der op til, at datalogi er et
specialiseret felt, som reserveres for dé, der tager universitetsuddennelsen
datalogi.

\section{Naur}
Naurs tekst er idealistist mens CSS er pragmatisk
CSS ligger vægt på nuværende uddannelse mens Naur kigger mod fremtiden.
Naur ligger mere fokus på teori
Mastery vs Data
Bør lære principper frem for sprog. Forståelsesfag.
ACM fokuserer i højere grad end Naurs tidlige tekster på udvikling og datamater, knap så
meget på den teoretiske fundering.
Kunne Naurs theory building / kreativitet være udspruget af, at han har været vant til, som astronom, at skulle
'sidde lidt i sin egen verden og filosofere'
Peter Naur fokuserer meget på at gruppearbejde og projektarbejde er vigtige elementer af en datalogisk uddannelse,
idet det underbygger forståelsen for hvad det vil sige at skulle udvikle et system.


\section{I vore dage}
ACM: Datalogi er vokset, derfor skal den gøres mere fleksibel.
        - datalogien skal kunne tilpasse sig den hastige udvikling - være omstillignsparat.
Idag er computeren et vigtigt værktøj i næsten samtlige grene af videnskaben,
softwareudvikling er blevet et key-element til at understøtte alle disse nye
systemer.
Meget software(udvikling) kræver en specialiseret viden (simuleringer etc.)
CS2013 siger at programmering skalvære en menneskelig aktivitet og ingeniøren skal kunne videreformidle sine systemer til ikke-teknisk kyndige.

\section{Udannelse og almen viden}
CSS fokuserer på software, Naur på data
Mindske grøften ved at uddanne, mens ?teori? kun gavner specialister
Projektledelse/kommunikation i universitetet vs lære om det i folkeskolen.
I folkeskolen tilføjes et nyt element (EDB) uden en egentlig datalogisk uddannelse

Han forsøgte at udbrede datalogien og gøre den (uddannelsen) tilgængelig (for alle).
 - den har været reserveret for specialister.
I CS2013 er der, i god stil med Peter Naurs tanker, en god del forståelsesfag.

Dét, at vi er en konstant udvikling gør, iflg Naur, at vi ikke bør fokusere på teknologier/værktøjer (inkl. datamater)

\section{Perspektivering}
Udviklingen er gået modsat hvad Naur ønskede. Mere computer, mindre computing

