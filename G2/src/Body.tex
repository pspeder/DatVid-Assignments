\section{Redegørelse}
Den nuværende videnskabelige metode går ud på at opsætte en hypotese, udføre eksperimenter, observere hvad der sker og bagefter sammenholde resultaterne med dem fra hypotesen. Ud fra det kan man verificere eller falsificere sin hypotese og så det kan man enten opstille en generel teori over emnet eller man kan revidere sin teori, alt efter, i hvor stor grad, ens resultater passer til hypotesen. 

\subsection{``End of Theory'' beskrivelse}
End of Theory er en artikel, der beskriver en ny tendens inden for videnskab. Den handler om at analysere store mængder data med matematik og fra de resultater man har fået så drager man sine konklusioner. Denne type "forskning" adskiller sig fra den normale praksis ved at den ikke bruger modeller og at man ikke behøver at opstille teorier. Man behøver ikke at vide hvorfor tingene sker, man ved bare at de sker. Denne praksis er muliggjort på grund af den store mængde data der bliver opsamlet af blandt andet Google og de store petabyte lagere. Data er ikke længere noget man visualisere, men er i stedet noget man ser igennem matematik hvormed korrelationen bliver tydelig.

\subsection{Modeller}
Vi vil her kort beskrive forskellen imellem den videnskabelige verdens
forståelse af en model og Andersons udlægning af, og specielt hans sprogbrug
omkring, denne.

\subsubsection{Hvad er en model?}
Wikipediaartiklen om videnskabelige modeller siger om emnet, det følgende:
\begin{quote}
Scientific modelling is a scientific activity the aim of which is to make a
particular part or feature of the world easier to understand, define, quantify, visualize, or simulate.\footnote{Kilde: {\url{http://en.wikipedia.org/wiki/Scientific_modelling}} ll. 1-3}
\end{quote}

Under denne definition er Andersons sprogbrug omkring modeller forkert, idet
hans forslag til, hvordan videnskab skal behandles, netop stræber efter ikke at
søge en ``\ldots particular part or feature of the world\ldots.''\\
Hans filosofi går netop ud på, ikke at have en forudindtaget opfattelse omkring
verden og om det, man ønsker at undersøge (hvis man da ved dét).

\subsubsection{Hvorfor modelleres der i videnskaben?}
Der har tidligere været et behov for måder at repræsentere virkeligheden på.
Modeller giver videnskabsmanden muligheden for at afprøve sin hypotese, selv
under omstændigheder, hvor direkte måling og eksperimentering er umulig.\\
Sådanne områder inkluderer blandt andre astronomien og har tidligere inkluderet
mange flere.

Modeller var så at sige, den metode, der mest konsistent tilnærmelsesvis beskrev
verden.

\subsubsection{Overvejelse}
Matematikken har altid adskilt sig fra den øvrige naturvidenskab, idet dens
videnskabsteori og metodebrug har fundamentale forskelle.\\
Der er i matematik en indbygget formalitet, der ikke i samme grad beskæftiger
sig med empiri. Denne formalitet vil naturligt integrere sig i resten af
naturvidenskaben, hvis et paradigmeskift, som Anderson foreslår det, skulle
indtræffe.

Hvis resten af naturvidenskaben fremover skulle basere sig på matematik, vil
de, i matematikken indbyggede paradokser, gøre hele fundamentet for
datamining, på en sådan skala, usikkert.\\
Der vil så muligvis blive mangel på et videnskabsparadigme, der kan arbejde
med sådanne usikkerheder. Her vil man se, at et sådan system allerede
eksisterer -- nemlig det nuværende, der baserer sig på modeller.
