\section{Konklusion}
Selvom Andersons ræsonment for hvorfor man ikke burde bruge modeller, 
teorier og hypoteser, og i stedet for finde korrelationer imellem data er 
mildest talt mangelfuld, kan man ikke sige at han tager fuldkommen fejl. 
Nogle gange behøver man ikke finde kasualiteten for dataen, hvor korrelationen 
er lige så dyrebar, hvis ikke mere. Men ikke alle situationer kræver den samme 
fremgangsmåde. Hvad fungerer for google ads ville måske ikke kunne fungere for 
noget helt andet. Og hans radikale mening om at psykologi, sociologi og 
taxodomi kunne blive afskaffet da man burde være ligeglad med hvorfor 
mennesker opfører sig som de gør kan man nærmest sige at værende fornærmende 
for videnskaben.\\\\

Som mange andre teknikker som bruges i forskellige dele af verdenen og videnskaben, 
er en plads til Andersons fremgang, som kan blive misfortolket grundet Andersons højt 
uvidenskabsteoretiske tilgang til videnskaben. Data er vejen frem, og den korrekte 
måde af håndtering af data er dyrebart, så længe den rette metode og fremgang benyttes.
